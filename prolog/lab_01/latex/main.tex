\documentclass[12pt]{report}
\usepackage[utf8]{inputenc}
\usepackage[russian]{babel}
%\usepackage[14pt]{extsizes}
\usepackage{listings}
\usepackage{graphicx}
\usepackage{amsmath,amsfonts,amssymb,amsthm,mathtools} 
\usepackage{pgfplots}
\usepackage{filecontents}
\usepackage{float}
\usepackage{comment}
\usepackage{indentfirst}
\usepackage{eucal}
\usepackage{enumitem}
%s\documentclass[openany]{book}
\frenchspacing

\usepackage{indentfirst} % Красная строка

\usetikzlibrary{datavisualization}
\usetikzlibrary{datavisualization.formats.functions}

\usepackage{amsmath}


% Для листинга кода:
\lstset{ %
	language=c,                 % выбор языка для подсветки (здесь это С)
	basicstyle=\small\sffamily, % размер и начертание шрифта для подсветки кода
	numbers=left,               % где поставить нумерацию строк (слева\справа)
	numberstyle=\tiny,           % размер шрифта для номеров строк
	stepnumber=1,                   % размер шага между двумя номерами строк
	numbersep=5pt,                % как далеко отстоят номера строк от подсвечиваемого кода
	showspaces=false,            % показывать или нет пробелы специальными отступами
	showstringspaces=false,      % показывать или нет пробелы в строках
	showtabs=false,             % показывать или нет табуляцию в строках
	frame=single,              % рисовать рамку вокруг кода
	tabsize=2,                 % размер табуляции по умолчанию равен 2 пробелам
	captionpos=t,              % позиция заголовка вверху [t] или внизу [b] 
	breaklines=true,           % автоматически переносить строки (да\нет)
	breakatwhitespace=false, % переносить строки только если есть пробел
	escapeinside={\#*}{*)}   % если нужно добавить комментарии в коде
}


\usepackage[left=2cm,right=2cm, top=2cm,bottom=2cm,bindingoffset=0cm]{geometry}
% Для измененных титулов глав:
\usepackage{titlesec, blindtext, color} % подключаем нужные пакеты
\definecolor{gray75}{gray}{0.75} % определяем цвет
\newcommand{\hsp}{\hspace{20pt}} % длина линии в 20pt
% titleformat определяет стиль
\titleformat{\section}[hang]{\Huge\bfseries}{\thechapter\hsp\textcolor{gray75}{|}\hsp}{0pt}{\Huge\bfseries}


% plot
\usepackage{pgfplots}
\usepackage{filecontents}
\usetikzlibrary{datavisualization}
\usetikzlibrary{datavisualization.formats.functions}

\begin{document}
	%\def\sectionname{} % убирает "Глава"
	\thispagestyle{empty}
	\begin{titlepage}
		\noindent \begin{minipage}{0.15\textwidth}
			\includegraphics[width=\linewidth]{b_logo}
		\end{minipage}
		\noindent\begin{minipage}{0.9\textwidth}\centering
			\textbf{Министерство науки и высшего образования Российской Федерации}\\
			\textbf{Федеральное государственное бюджетное образовательное учреждение высшего образования}\\
			\textbf{~~~«Московский государственный технический университет имени Н.Э.~Баумана}\\
			\textbf{(национальный исследовательский университет)»}\\
			\textbf{(МГТУ им. Н.Э.~Баумана)}
		\end{minipage}
		
		\noindent\rule{18cm}{3pt}
		\newline\newline
		\noindent ФАКУЛЬТЕТ $\underline{\text{«Информатика и системы управления»}}$ \newline\newline
		\noindent КАФЕДРА $\underline{\text{«Программное обеспечение ЭВМ и информационные технологии»}}$\newline\newline\newline\newline\newline
		
		\begin{center}
			\noindent\begin{minipage}{1.1\textwidth}\centering
				\Large\textbf{Отчет по лабораторной работе №7}\newline
				\textbf{по дисциплине <<Функциональное и логическое}\newline
				\textbf{~~~программирование>>}\newline\newline
			\end{minipage}
		\end{center}
		
		\noindent\textbf{Тема} $\underline{\text{Среда Visual Prolog.}}$\newline\newline
		\noindent\textbf{Студент} $\underline{\text{Турчанинов А. М.~~~~~~~~~~~~~~~~~~~~~~~~~~~~~~~~~~~~~~~~~~~~~~~~~~~~~~~~~~~~~~~~~}}$\newline\newline
		\noindent\textbf{Группа} $\underline{\text{ИУ7-65Б~~~~~~~~~~~~~~~~~~~~~~~~~~~~~~~~~~~~~~~~~~~~~~~~~~~~~~~~~~~~~~~~~~~~~~~~~}}$\newline\newline
		\noindent\textbf{Оценка (баллы)} $\underline{\text{~~~~~~~~~~~~~~~~~~~~~~~~~~~~~~~~~~~~~~~~~~~~~~~~~~~~~~~~~~~~~~~~~~~~~~~~}}$\newline\newline
		\noindent\textbf{Преподаватели} $\underline{\text{Толпинская Н.Б., Строганов Ю. В.~~~~~~~~~~~~~~~~~~~~~~~~~~}}$\newline\newline\newline
		
		\begin{center}
			\vfill
			Москва~---~\the\year
			~г.
		\end{center}
	\end{titlepage}

\section*{Задание}
Запустить среду Visual Prolog5.2. Настроить утилиту TestGoal.

Запустить тестовую программу, проанализировать реакцию системы и множество ответов.

Разработать свою программу - «Телефонный справочник». Абоненты могут иметь несколько телефонов. Протестировать работу программы, используя разные вопросы. 

\begin{itemize}
    \item[--] «Телефонный справочник»: Фамилия, №тел, Адрес – структура (Город, Улица, №дома,
    №кв).
    \item[--] «Автомобили»: Фамилия владельца, Марка, Цвет, Стоимость, Номер.
    \item[--] Владелец может иметь несколько телефонов, автомобилей (Факты). В разных городах есть однофамильцы, в одном городе – фамилия уникальна.
\end{itemize}

Используя конъюнктивное правило и простой вопрос, обеспечить возможность поиска:

\begin{itemize}
	\item[--] По Марке и Цвету автомобиля найти Фамилию, Город, Телефон . Лишней информации не находить и не передавать!!!
\end{itemize}

\chapter*{Решение}
\begin{lstlisting}
domains
lastName, phone = symbol.
model, color = symbol.
city, street, house, flat = symbol.
number, price = integer.
address = address(city, street, house, flat).

predicates

tel(lastName, phone, address).
car(lastName, model, color, price, number).
person(lastName, model, color, phone, city).

clauses

person(LastName, Model, Color, Phone, City):- tel(LastName, Phone, address(City, _, _, _)), car(LastName, Model, Color, _, _).

tel("Petrov", "812314214", address("Moscow", "Baumana", "10", "4")).
tel("Ivanov", "817314214", address("Moscow", "Baumana", "12", "7")).
tel("Nickolaev", "815314214", address("Moscow", "Baumana", "15", "2")).
tel("Nickolaev", "815314214", address("Saratov", "Baumana", "15", "2")).
tel("Nickolaev", "812456431", address("Saratov", "Baumana", "15", "2")).

car("Petrov", "Mercedes", "black", 10000000, 453).
car("Ivanov", "Mercedes", "yellow", 15000000, 536).
car("Nickolaev", "Toyota", "black", 15000000, 154).

goal
person(LastName, "Mercedes", "yellow", Tel, City).
person(LastName, _, "black", Tel, City).
person(LastName, "Mercedes", _, Tel, City).
tel("Ivanov", Phone, _).
car("Nickolaev", _, Color, _, _).
person(_, Model, Color, _, "Moscow").
person(LastName, _, "black", _, _).
person("Petrov", Model, _, _, _).
person(_, _, _, _, address("Moscow", "Baumana", _, _)).
car(_, _, _, _, 536), tel(LastName, Phone, _).
car(_, _, _, _, 453), tel(LastName, _, address(_, _, _, _)).
car("Nickolaev", _, _, Price, _).
car("Petrov", _, "black", _, Number).
\end{lstlisting}

Покажем результаты тестирования нашей программы:
\begin{enumerate}
	\item \texttt{person(LastName, "Mercedes", "yellow", Tel, City)}. \\
	\texttt{LastName = "Ivanov", Tel = "817314214", City = "Moscow"}
	\item \texttt{person(LastName, \_, "black", Tel, City)}. \\
	\texttt{LastName = "Petrov", Tel = "812314214", City = "Moscow"} \\
	\texttt{LastName = "Nickolaev", Tel = "815314214", City = "Moscow"}
	\item \texttt{person(LastName, "Mercedes", \_, Tel, City)}. \\
	\texttt{LastName = "Ivanov", Tel = "817314214", City = "Moscow"} \\
	\texttt{LastName = "Petrov", Tel = "812314214", City = "Moscow"}
	\item \texttt{tel("Ivanov", Phone, \_)}. \\
	\texttt{Phone = "817314214"}
	\item \texttt{car("Nickolaev", \_, Color, \_, \_)}. \\
	\texttt{Color = "yellow"}
	\item \texttt{person(\_, Model, Color, \_, "Moscow")}. \\
	\texttt{Model = "Mercedes", Color = "yellow"} \\
	\texttt{Model = "Toyota", Color = "black"}
	\item \texttt{person(LastName, \_, "black", \_, \_)}. \\
	\texttt{LastName = "Nickolaev"} \\
	\texttt{LastName = "Petrov"}
	\item \texttt{person("Petrov", Model, \_, \_, \_)}. \\
	\texttt{Model = "Mercedes"}
	\item \texttt{person(\_, \_, \_, \_, address("Moscow", "Baumana", \_, \_))}. \\
	\texttt{True} (может быть несколько повторений)
	\item \texttt{car(\_, \_, \_, \_, 536), tel(LastName, Phone, \_)}. \\
	\texttt{LastName = "Ivanov", Phone = "817314214"}
	\item \texttt{car(\_, \_, \_, \_, 453), tel(LastName, \_, address(\_, \_, \_, \_))}. \\
	\texttt{LastName = "Petrov"}
	\item \texttt{car("Nickolaev", \_, \_, Price, \_)}. \\
	\texttt{Price = 15000000}
	\item \texttt{car("Petrov", \_, "black", \_, Number)}. \\
	\texttt{Number = 453}
\end{enumerate}


	


\end{document}