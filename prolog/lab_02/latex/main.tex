\documentclass[12pt]{report}
\usepackage[utf8]{inputenc}
\usepackage[russian]{babel}
%\usepackage[14pt]{extsizes}
\usepackage{listings}
\usepackage{graphicx}
\usepackage{amsmath,amsfonts,amssymb,amsthm,mathtools} 
\usepackage{pgfplots}
\usepackage{filecontents}
\usepackage{float}
\usepackage{comment}
\usepackage{indentfirst}
\usepackage{eucal}
\usepackage{enumitem}
%s\documentclass[openany]{book}
\frenchspacing

\usepackage{indentfirst} % Красная строка

\usetikzlibrary{datavisualization}
\usetikzlibrary{datavisualization.formats.functions}

\usepackage{amsmath}


% Для листинга кода:
\lstset{ %
	language=c,                 % выбор языка для подсветки (здесь это С)
	basicstyle=\small\sffamily, % размер и начертание шрифта для подсветки кода
	numbers=left,               % где поставить нумерацию строк (слева\справа)
	numberstyle=\tiny,           % размер шрифта для номеров строк
	stepnumber=1,                   % размер шага между двумя номерами строк
	numbersep=5pt,                % как далеко отстоят номера строк от подсвечиваемого кода
	showspaces=false,            % показывать или нет пробелы специальными отступами
	showstringspaces=false,      % показывать или нет пробелы в строках
	showtabs=false,             % показывать или нет табуляцию в строках
	frame=single,              % рисовать рамку вокруг кода
	tabsize=2,                 % размер табуляции по умолчанию равен 2 пробелам
	captionpos=t,              % позиция заголовка вверху [t] или внизу [b] 
	breaklines=true,           % автоматически переносить строки (да\нет)
	breakatwhitespace=false, % переносить строки только если есть пробел
	escapeinside={\#*}{*)}   % если нужно добавить комментарии в коде
}


\usepackage[left=2cm,right=2cm, top=2cm,bottom=2cm,bindingoffset=0cm]{geometry}
% Для измененных титулов глав:
\usepackage{titlesec, blindtext, color} % подключаем нужные пакеты
\definecolor{gray75}{gray}{0.75} % определяем цвет
\newcommand{\hsp}{\hspace{20pt}} % длина линии в 20pt
% titleformat определяет стиль
\titleformat{\section}[hang]{\Huge\bfseries}{\thechapter\hsp\textcolor{gray75}{|}\hsp}{0pt}{\Huge\bfseries}


% plot
\usepackage{pgfplots}
\usepackage{filecontents}
\usetikzlibrary{datavisualization}
\usetikzlibrary{datavisualization.formats.functions}

\begin{document}
	%\def\sectionname{} % убирает "Глава"
	\thispagestyle{empty}
	\begin{titlepage}
		\noindent \begin{minipage}{0.15\textwidth}
			\includegraphics[width=\linewidth]{b_logo}
		\end{minipage}
		\noindent\begin{minipage}{0.9\textwidth}\centering
			\textbf{Министерство науки и высшего образования Российской Федерации}\\
			\textbf{Федеральное государственное бюджетное образовательное учреждение высшего образования}\\
			\textbf{~~~«Московский государственный технический университет имени Н.Э.~Баумана}\\
			\textbf{(национальный исследовательский университет)»}\\
			\textbf{(МГТУ им. Н.Э.~Баумана)}
		\end{minipage}
		
		\noindent\rule{18cm}{3pt}
		\newline\newline
		\noindent ФАКУЛЬТЕТ $\underline{\text{«Информатика и системы управления»}}$ \newline\newline
		\noindent КАФЕДРА $\underline{\text{«Программное обеспечение ЭВМ и информационные технологии»}}$\newline\newline\newline\newline\newline
		
		\begin{center}
			\noindent\begin{minipage}{1.1\textwidth}\centering
				\Large\textbf{Отчет по лабораторной работе №8}\newline
				\textbf{по дисциплине <<Функциональное и логическое}\newline
				\textbf{~~~программирование>>}\newline\newline
			\end{minipage}
		\end{center}
		
		\noindent\textbf{Тема} $\underline{\text{Среда Visual Prolog. Структура программы. Работа программы.}}$\newline\newline
		\noindent\textbf{Студент} $\underline{\text{Турчанинов А. М.~~~~~~~~~~~~~~~~~~~~~~~~~~~~~~~~~~~~~~~~~~~~~~~~~~~~~~~~~~~~~~~~~}}$\newline\newline
		\noindent\textbf{Группа} $\underline{\text{ИУ7-65Б~~~~~~~~~~~~~~~~~~~~~~~~~~~~~~~~~~~~~~~~~~~~~~~~~~~~~~~~~~~~~~~~~~~~~~~~~}}$\newline\newline
		\noindent\textbf{Оценка (баллы)} $\underline{\text{~~~~~~~~~~~~~~~~~~~~~~~~~~~~~~~~~~~~~~~~~~~~~~~~~~~~~~~~~~~~~~~~~~~~~~~~}}$\newline\newline
		\noindent\textbf{Преподаватели} $\underline{\text{Толпинская Н.Б., Строганов Ю. В.~~~~~~~~~~~~~~~~~~~~~~~~~~}}$\newline\newline\newline
		
		\begin{center}
			\vfill
			Москва~---~\the\year
			~г.
		\end{center}
	\end{titlepage}

\section*{Задание}
Создать базу знаний «Собственники», дополнив (и минимально изменив) базу знаний, хранящую знания:
\begin{itemize}
    \item «Телефонный справочник»: Фамилия, Noтел, Адрес – структура (Город, Улица, Noдома, Noкв),
    \item «Автомобили»: Фамилия\_владельца, Марка, Цвет, Стоимость, и др.,
    \item «Вкладчики банков»: Фамилия, Банк, счет, сумма, др.
\end{itemize}
знаниями о дополнительной собственности владельца. Преобразовать знания об автомобиле к форме знаний о собственности. Вид собственности (кроме автомобиля):
\begin{itemize}
    \item Строение, стоимость и другие его характеристики;
    \item Участок, стоимость и другие его характеристики;
    \item Водный\_транспорт, стоимость и другие его характеристики.
\end{itemize}

Описать и использовать вариантный домен: Собственность. Владелец может иметь, но только один объект каждого вида собственности (это касается и автомобиля), или не иметь некоторых видов собственности.

Используя конъюнктивное правило и разные формы задания одного вопроса (пояснять для какого No задания – какой вопрос), обеспечить возможность поиска:
\begin{enumerate}
    \item Названий всех объектов собственности заданного субъекта,
    \item Названий и стоимости всех объектов собственности заданного субъекта,
    \item * Разработать правило, позволяющее найти суммарную стоимость всех
объектов собственности заданного субъекта.
\end{enumerate}

Для 2-го пункт и одной фамилии составить таблицу, отражающую конкретный
порядок работы системы, с объяснениями порядка работы и особенностей использования доменов (указать конкретные Т1 и Т2 и полную подстановку на каждом шаге)

\chapter*{Решение}
\begin{lstlisting}
domains
name, phone, univer, color, bankName, city, street, house, flat = string.
amount, price = integer.
address = address(city, street, house, flat).
object = building(name, price);
region(name, price);
water_transport(name, color, price);
car(name, color, price).

predicates
tel(name, phone, address).
deposit(name, bankName, amount).
owner(name, object).

all_objects(name, name).
all_objects_price(name, name, price).
all_objects_price_sum(name, price).
all_objects_price_sum_elem(name, symbol, price).

clauses
all_objects_price(Surname, Name, Price) :- owner(Surname, car(Name, _, Price)).
all_objects_price(Surname, Name, Price) :- owner(Surname, building(Name, Price)).
all_objects_price(Surname, Name, Price) :- owner(Surname, region(Name, Price)).
all_objects_price(Surname, Name, Price) :- owner(Surname, water_transport(Name, _, Price)).

owner("Denis", car("BMW", "Green", 1000)).
owner("Egor", region("Nothung", 0)).
owner("Darya", building("Moscow-city", 100500)).
owner("Valera", car("BMW", "green", 1000)).
owner("Anton", region("Krasnogorsk", 10000)).
owner("Denis", building("Mail.ru Office", 20000)).
owner("Egor", water_transport("Yacht", "Red", 10000)).
owner("Darya", car("Cadillac", "Black", 304000)).
owner("Anton", building("BMSTU", 200000)).
owner("Valera", car("Mercedes", "White", 30000)).

all_objects(Surname, Name) :- owner(Surname, car(Name, _, _)).
all_objects(Surname, Name) :- owner(Surname, building(Name, _)).
all_objects(Surname, Name) :- owner(Surname, region(Name, _)).
all_objects(Surname, Name) :- owner(Surname, water_transport(Name, _, _)).

all_objects_price_sum_elem(Surname, building, Price):- owner(Surname, building(_, Price)).
all_objects_price_sum_elem(Surname, region, Price):- owner(Surname, region(_, Price)).
all_objects_price_sum_elem(Surname, water_transport, Price):- owner(Surname, water_transport(_, _, Price)).
all_objects_price_sum_elem(Surname, car, Price):- owner(Surname, car(_, _, Price)).
all_objects_price_sum_elem(_, _, 0).

all_objects_price_sum(Surname, Price):- all_objects_price_sum_elem(Surname, building, Price1), all_objects_price_sum_elem(Surname, region, Price2),
all_objects_price_sum_elem(Surname, water_transport, Price3), all_objects_price_sum_elem(Surname, car, Price4), Price = Price1 + Price2 + Price3 + Price4.

tel("Anton", "812314214", address("moscow", "olenevaya", "12", "4")).
tel("Egor", "814314214", address("moscow", "olenevaya2", "12", "4")).
tel("Denis", "815314214", address("moscow", "olenevaya3", "12", "4")).
tel("Darya", "815314214", address("moscow", "olenevaya3", "12", "4")).
tel("Darya", "817314214", address("moscow", "olenevaya", "13", "4")).
tel("Valera", "816314214", address("moscow", "olenevaya2", "16", "4")).

deposit("Egor", "sber", 1000).
deposit("Anton", "tinkoff", 20000).
deposit("Denis", "raif", 100000).
deposit("Valera", "sber", 10000).

goal
%all_objects("Denis", X);
%all_objects_price("Denis", X, Y).
all_objects_price_sum("Denis", X).
\end{lstlisting}

\pagebreak
Порядок формирования результата для 2-го вопроса:

\begin{table}[H]
	\begin{center}
		\begin{tabular}{|c c c |} 
			\hline
			№ шага & Сравниваемые термы; результаты; подстановка, если есть & Дальнейшие действия \\  
			\hline
			1 & Сравниваются & Прямой ход \\
			  & all\_objects\_price(Surname, Name, Price) & \\
			  & и all\_objects\_price(<<Denis>>, X, Y)  & \\
			  & Подстановка (Surname - <<Denis>>) &\\
			\hline
			2-5 & Сравниваются owner(<<Denis>> , car(Name, \_, Price)) & Термы не \\
			  & и all\_objects\_price(Surname, Name, Price) & унифицируемы. \\
			  & Они имеют разные функторы. &Переход к следующему \\
			  & & предложению\\		
			\hline
			6 & Сравниваются  owner(<<Denis>>, car(Name, \_, Price)) & Прямой ход. \\
			  & и owner(<<Denis>>, car(<<BMW>>, <<Green>>,  & Занесение \\
			  & 1000)). Подстановка & Name = <<BMW>>{} \\
			  & (Name = <<BMW>>, \_ = <<Green>>{}, & Price = 1000\\
			  & Price = 1000) & в ячейку\\
			\hline
			7-30 & Сравниваются owner(<<Denis>>, car(Name, \_, Price)) & Термы не \\
			  & и ... & унифицируемы. \\
		      &  Термы   & Переход к следующему \\
			  & не унифицируемы. & предложению. \\
			
			\hline
			31 & Сравниваются & Прямой ход \\
			  & all\_objects\_price(Surname, Name, Price) & \\
			  & и all\_objects\_price(<<Denis>>, X, Y)  & \\
			  & Подстановка (Surname - <<Denis>>) &\\
			\hline
			32-35 & Сравниваются owner(<<Denis>>, building(Name, Price)) & Термы не \\
			  & и all\_objects\_price(Surname, Name, Price) & унифицируемы. \\
			  & Они имеют разные функторы. &Переход к следующему \\
			  & & предложению\\		
			\hline
			36-40 & Сравниваются  owner(<<Denis>>, building(Name, Price)) & Прямой ход. \\
			  & и ... & унифицируемы. \\
		      &  Термы   & Переход к следующему \\
			  & не унифицируемы. & предложению. \\
			\hline
			41 & Сравниваются owner(<<Denis>>, building(Name, Price)) & Прямой ход. \\
			  & и owner(<<Denis>>, building(<<Mail.ru Office>>, & Занесение \\
			  &  20000)). Подстановка & Name = <<Mail.ru Office>> \\
			  & (Name = <<Mail.ru Office>>, & Price = 20000\\
			  & Price = 20000) & в ячейку\\

			\hline
			42-59 & Сравниваются owner(<<Denis>>, building(Name, Price))) & Термы не \\
			  & и ... & унифицируемы. \\
			  & Они имеют разные функторы. &Переход к следующему \\
			  & & предложению\\
\hline
\end{tabular}
	\end{center}
\end{table}

\end{document}